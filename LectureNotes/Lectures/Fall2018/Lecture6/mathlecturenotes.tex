\documentclass[12pt]{article}

\usepackage[utf8]{inputenc} %For non-ASCII characters%
\usepackage[margin=1in]{geometry} %For margins and related things%
\usepackage{amssymb, amsthm, amsmath, bbm}

\let\oldphi\phi
\renewcommand{\phi}{\varphi}

\newenvironment{soln}{\begin{proof}[\textsc{Solution}]}{\renewcommand{\qedsymbol}{$\blacklozenge$}\end{proof}}

\title{Poisson Point Processes Cont.}
\author{Erin McCarthy\thanks{\textbf{email:} emccarth@uoregon.edu}}
\date{\today}
\begin{document}
\maketitle

\section{The Cauchy Distribution}

$C \sim Cauchy(1)$ 
if $\mathbb{P}\{c \in dx\}= \dfrac{dx}{\pi(1+x)^2}$ and $t \times C \sim Cauchy(t)$

\noindent\rule{4cm}{0.4pt} 

\noindent$N \sim PPP(\dfrac{dtdx}{|x|^2})$ on $[0,\infty) \times \{ \mathbb{R}\setminus \{0\}\}$

\noindent$Let C_t = ($sum of x-values of points of $[0,t] \times \mathbb{R})= \sum x \mathbbm{1}_{[0,t)}(t) dN(t,x)$ 

\subsection*{Claim}
$\dfrac{1}{t} C_t \stackrel{d}{=} C_1$

\begin{proof}
If $N = \sum \delta_{(t_i,x_i)} \Rightarrow \dfrac{1}{t} C_t = \sum_{i:t_i\leq1} \dfrac{x_i}{t}$ but \~N $= \sum_i \delta_{\left(\frac{t_i}{t_1}, \frac{x_i}{t}\right)} \sim PPP(\dfrac{dtdx}{x^2})$ i.e. $N \stackrel{d}{=} $\~N
(since $\dfrac{dsdx}{x^2} = \dfrac{tdu tdv}{t^2dv}=\dfrac{dudv}{v^2}$, where $u=\frac{s}{t}$ and $v=\frac{x}{t}$), so $\dfrac{1}{t}C_t = \int x \mathbbm{1}_{[0,1]}(t) d\text{\~{N}} (t,x) = \text{\~C}(1)$ 
\end{proof}


\subsection*{Consequence}
Let $x_1,x_2,...,x_n$ iid $Cauchy(1)$ then $\bar{x} = \dfrac{1}{n} (x_1+...+x_n) \sim Cauchy(1)$. This implies that that taking more samples does not tell you more about the mean! But ...\\
$$\begin{aligned}
var[\bar{x}] = \frac{1}{n}var[x_1] = \int_0^1 \int_{-\infty}^{\infty} x^2 \frac{dxdt}{x^2} = \infty \\ \text{ even } \mathbb{E}[x_1]=\int_0^1\int_{-\infty}^\infty x \frac{dxdt}{x^2} = \text{ UND }\\
\end{aligned}$$



\subsection*{Question:}
Suppose we have arrival times on $[0,\infty).$ Are they (uniform) Poisson?

Let $0< T_1 <T_2<..$ be the points, compare the distribution of $(T_{k+1}-T_k)^{N}_{k=1}$ inter-arrival intervals to the exponential distribution. e.g by Kolmogorov-Smirnov test (say that 10 times fast).

We discussed two approaches to this problem in class including: looking the the length between the points or sectioning it into areas to examine the distribution.

\subsection*{Problem: London bombings} Let ${x_1,...,x_m}$ be points in the (2D) torus, T. Are they from a $PPP(m\times\lambda)$ i.e. constant intensity M?\\
\textbf{Approach:} look for over-or under- dispersal.
\begin{enumerate}
    \item Estimate $m$: if $M \sim Pois(m) \Rightarrow \text{then } M \approx m$
    
    \item Estimate local density: 
    
    Let $\rho(r):\mathbb{R}_{\geq 0}\rightarrow \mathbb{R}_{\geq 0}$, where $\rho(r)=0$ if $n>\frac{1}{2}$, $\rho(r) \geq 0$, and $\int_{T} \rho(|x|)dx=1$.
    
    When $N=\sum_i \delta {x_i}$ and  $N\sim PPP(m)$, let $P_i = \sum\limits_{j\neq i} \rho (|x_j-x_i|)$ and $P=\sum_i P_i =\int_{T\times T\setminus diag(T)} e^{(|x-y|)}N(dx)N(dy)$. Then:

    $$\begin{aligned}
    \mathbb{E}[N] &= \mathbb{E}\int_{T\times T \setminus diag(t)}  \rho (|x-y|)N(dx)N(dy) \\
    &= \int_{T\times T \setminus diag(T)} \rho  (|x-y|) \mathbb{E}[N(dx)N(dy)] \\
    &= \lim_{\varepsilon \to 0} \frac{1}{(\pi \varepsilon^2)}^2 N(\beta_\varepsilon(x))N(\beta_\varepsilon(y)) 
    \dfrac{1}{\varepsilon^4\pi^2}(\mathbb{E}[N(\beta_\varepsilon(x)N(\beta_\varepsilon(y)] \\ 
    &= \mathbb{E}[N(\beta_\varepsilon(x)]\mathbb{E}[N_\varepsilon(y)]+ cov[N(\beta_\varepsilon(x)), N(\beta_\varepsilon(y)])\\
    & \xrightarrow[]{\varepsilon \searrow 0} m^2dxdy, \text{(if } x \neq y)
    \end{aligned}$$
\end{enumerate}

\noindent Thus, under the conditions of step 2, when $N\sim PPP(m)$, we have that
$$\begin{aligned}
\mathbb{E}[P] &= \int_{T\times T \setminus diag(T)} \mathbb{E}[N(dx)N(dy)] \rightarrow \int_T \int_T P(|x-y|)m^2 dxdy\\
 &= \int_T m^2 dx = m^2
\end{aligned}$$

\noindent So: $\frac{P}{m^2}\approx 1 $ for a uniform PPP.

%\subsection*{Recall:} We have points ${x_1,...,x_m}$ estimate of local density around $x_i$ on T. Are they Poisson?
%\noindent\textbf{Strategy:} 
%
%$$\begin{aligned}
%\text{Let } P_i = & \sum_{j\neq i} \rho (|x_i -x_j|) \text{ and }  P = \sum_i P_i \text{ for } e(r) \text{ in } \int_T\rho(|r|)dr=1\\
%\text{ If } N\sim  PPP&  \text{ on } T \text{ with uniform intensity } m \text{ then } 
%\mathbb{E}[P] = \int_{T\times T \setminus diag(T)} \mathbb{E}[N(dx)N(dy)] \\
%\text{ and } \mathbb{E}&[N(dx)N(dy)] = m^2dxdy \text{ for } x\neq y\\
%\end{aligned}$$

\subsection*{Aside:}
In general, if it is not a Poisson process in the above equations we would have that: 
$$\begin{aligned}
\mathbb{E}[N(dx)N(dy)] = m^2dxdy +  cov[N(dx),N(dy)] = m^2g(x,y)dxdy
\end{aligned}$$

where $g$ is the correlation function such that:
$g(x,y) = \lim_{\varepsilon \to 0} \dfrac{1}{\pi^2\varepsilon^4}cor[\beta_\varepsilon(x),\beta_\varepsilon(x_2)]$. Then,

$$\begin{aligned}
\int_{T\times T \setminus diag(T)}\mathbb{E}[N(dx)N(dy)] &\rightarrow \int_Tm^2dx + \int_T\int_T\rho(|x-y|)g(x,y)m^2dxdy \\
&=  m^2 + \text{SOMETHING ELSE}
\end{aligned}$$

The ``SOMETHING ELSE" will be bigger if points tend to be clustered on the scale determined by $p$, and smaller if points repels each other on that scale.

\subsection*{How far off do we expect?}

$$\begin{aligned}
var[P] &= cov[P,P] \\
&= cov[\int_{T\times T \setminus diag(T)}\rho(|x-y|) N(dx)N(dy), \int_{T\times T \setminus diag(T)}\rho(|u-v|)N(du)N(dv)] \\ 
&= \int_A\rho(|x-y|)\rho(|u-v|) cov[N(dx)N(dy),N(du)N(dv)]
\end{aligned}$$

Let $A=(x,y,u,v) \in T^4 : x\neq y, u\neq v$. Then we have the following cases:

$$\begin{cases}
x\notin \{u,v\}, y\notin \{u,v\} &\Rightarrow cov[N(dx) N(dy), N(du)N(dv)] = 0\\
x=u, y\notin\{u,v\} 		  &\Rightarrow cov[-,-] = \mathbb{E}[N(dy)]\mathbb{E}[N(du)]var[N(dx)] = m^3dxdydv\delta_x(du)\\
x=u,y=v, (x\neq y)		  &\Rightarrow cov[-,-] = m^2dxdy \delta_x(du)\delta_y(dv)\\
\end{cases}$$

Which implies that:
\
$$\begin{aligned}
var[P] &= 4\int_T \int_T\int_T \rho (|x-y|) \rho(|x-v|)m^3 dxdydv + 2\int_T\int_T\rho (|x-y|)^2 m^2dxdy\\
	&= \mathbf{A} + \mathbf{B}
\end{aligned}$$

But what does that mean?

Let $\rho(r) = \begin{cases} \frac{1}{\pi R^2} \text{, if } r<R  \\  0, \text{  else.} \end{cases}$. Then $\mathbf{A} = 4m^3$ and $\mathbf{B}=\frac{2m^2}{\pi R^2}$.

So, $var[\frac{P}{M^2}] \approx \frac{4}{m} + \frac{2}{\pi R^2 m^2} $.

Thus, to determine if the bombs are being dropped at random (i.e. via a PPP), we can use the statistic:
$$\begin{aligned}
\frac{| \frac{P}{m^2} - 1 |} {\sqrt{ \frac{4}{m} + \frac{2}{\pi R^2 m^2} }} < 1.96 \Rightarrow \text{PPP}
\end{aligned}$$

\end{document}
